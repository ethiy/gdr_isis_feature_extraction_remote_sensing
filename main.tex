\documentclass[a4paper,french]{article}
    \usepackage{rfiap2018_resume}
    
    \usepackage[utf8]{inputenc}
    \usepackage[T1]{fontenc}
    \usepackage{babel}
    \usepackage{times, fourier}
    
    \usepackage{array}
    \newcolumntype{x}[1]{>{\centering\let\newline\\\arraybackslash\hspace{0pt}}p{#1}}
    \usepackage{tabulary}
    \usepackage{booktabs}
    \usepackage{multirow}
    
    
    \usepackage{siunitx}
    
    \usepackage{standalone}
    
    \usepackage{enumitem}
    
    \usepackage{amsmath}
    
    \usepackage{float, wrapfig}
    \usepackage[tablename=Tab.]{caption}
    \usepackage{color}
    
    \usepackage{etoolbox, setspace}
    \patchcmd{\thebibliography}
      {\settowidth}
      {\setlength{\itemsep}{0pt plus 0.1pt}\settowidth}
      {}{}
    \apptocmd{\thebibliography}
      {\small}
      {}{}
    
    
    \begin{document}
        \date{}
        \title{
            \Large\bf Qualification sémantique de modèles $3D$ de bâtiments
        }
        \author{
            \begin{tabular}[t]{c@{\extracolsep{4em}}c@{\extracolsep{4em}}c@{\extracolsep{4em}}c}
                Oussama Ennafii${}^1$ & Arnaud Le Bris${}^1$ & Florent Lafarge${}^2$ & Clément Mallet${}^1$ \\
            \end{tabular}
            {}\\
            \\
            ${}^1$        Univ. Paris-Est, LaSTIG MATIS, IGN, ENSG, 94160 Saint-Mandé, France\\
            ${}^2$        Inria, Titane, 06902 Sophia Antipolis, France
            {}\\
            \\
            oussama.ennafii@ign.fr\\
        }
        \maketitle
        \thispagestyle{empty}
        Bien que la reconstruction automatique de modèles $3D$ urbains à partir de données géospatiales soit communément utilisé, les opérateurs, dans un cadre de production, doivent évaluer visuellement, à l'échelle de la ville, la validité des modèles et détecter des erreurs de modélisation qui sont inévitables. Un tel processus repose sur des experts et prend beaucoup de temps (2 h/km$^2$/expert). Dans ce travail, nous proposons une approche pour évaluer automatiquement la qualité des modèles $3D$ de bâtiments. Les erreurs potentielles sont compilées dans une taxonomie hiérarchique, polyvalente et paramétrable. Cela permet pour la première fois de démêler les erreurs de fidélité et de modélisation, quel que soit le niveau de détail (LoD) des bâtiments modélisés. La qualité des modèles est prédite en utilisant les propriétés géométriques des bâtiments et, le cas échéant, des images géospatiales et des cartes de profondeur. Une Baseline d'attributs génériques et manuellement construits alimente un classificateur de forêt aléatoire. Les cas Multi-classes et Multi-labels sont pris en compte. En raison de l'indépendance entre les classes d'erreurs, nous avons la possibilité de retrouver toutes ces classes en prédisant les bâtiments erronés. En comparant les résultats de notre pipeline sur une plusieurs zones urbaines, nous pouvons détecter de manière satisfaisante, en moyenne, $96\%$ des erreurs les plus fréquentes.
        \begin{figure}[H]
          \begin{center}
              \includestandalone[mode=buildnew, width=\textwidth]{graphical_abstract}
             \caption{\label{fig::pipeline} Notre paradigme d'évaluation sémantique: en plus du modèle topologique illustré en (b), des attributs sont extraits en comparant avec des cartes de hauteur, comme représenté par la différence calculée entre la hauteur du modèle et le Modèle Numérique de Surface en (c). Les images géospatiales peuvent également être utilisées pour caractériser des modèles en comparant leurs projections à des gradients locaux (voir (d)). En se basant sur ces attributs, les erreurs sont prédites à l'aide d'un classificateur préétabli.}
          \end{center}
    	\end{figure}
        \bibliographystyle{abbrv}
        \begin{spacing}{0.01}
            \bibliography{references}
        \end{spacing}
    \end{document}
    